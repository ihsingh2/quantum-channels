\section{Accessible Information}

\begin{frame}{Accessible Information}
The Quantum no-cloning theorem establishes that quantum states cannot be perfectly cloned. This implies that there is an upper bound to the information that can be extracted from a quantum channel. The upper bound of information that is extractable from a quantum system is called the Holevo bound.\\
In classical systems, information is encoded deterministically, such as binary bits. In principle, all the encoded information can be accessed without limitations. In Quantum systems, however, information is encoded in quantum states, which can be superpositions or mixed states. This limits the amount of information that can be extracted from the system.
\end{frame}

\begin{frame}{Accessible Information}
Let a sender(Alice) encode a classical variable X into a quantum state $\rho_x$. The receiver(Bob) attempts to decode X by measuring the quantum state. The accessible information is defined as the maximum classical mutual information between X and the output decoded Y:
\[
I_{accessible} = \text{max } I(X;Y)
\]
Where the maximum is taken over all possible measurements that can  be performed, represented by POVMs
\end{frame}
