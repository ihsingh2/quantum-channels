\section{Super Additivity of Quantum Channels}

\begin{frame}{Superadditivity}
    Superadditivity in quantum information theory refers to the phenomenon where certain types of quantum capacities increase when multiple uses of a
    quantum channel are considered together, rather than treating each channel use independently. This property arises because quantum entanglement and
    correlations can enhance the effectiveness of the channel when used in a collective or joint manner.\\
    This superadditivity is the result of entanglement and other exotic correlation phenomena that arises in quantum systems. Superadditivity enhances
    communication capabilities of quantum channels and allows more communication than what their classical counterparts could (sometimes even through
    zero capacity channels).
\end{frame}

\begin{frame}{Superadditivity of various quantities}
    \begin{itemize}
        \item Coherent information of a quantum channel:
        $$Q(\mathcal{N} \otimes \mathcal{M} ) \geq Q(\mathcal{N} ) + Q(\mathcal{M} )$$
        or $$(Q(\mathcal{N}^{\otimes m}) > mQ(\mathcal{N}))$$
        where $Q(\mathcal{N}) = \max_{\phi_{AA'}}I(A\rangle B)_\rho$ and ${\phi_{AA'}}$ are all possible input states
        \item Holevo information:
        $$\chi (\mathcal{N} \otimes \mathcal{M} ) \geq \chi (\mathcal{N}) + \chi (\mathcal{M})$$
        \item Private capacity (over $m$ applications of the channel):
        $$P(\mathcal{N}^{\otimes m}) > mP(\mathcal{N})$$
    \end{itemize}    
\end{frame}

\begin{frame}
    Since Holevo information and Coherent information are superadditive, Classical capacity and Quantum capacity are, by extension, also superadditive.
\end{frame}