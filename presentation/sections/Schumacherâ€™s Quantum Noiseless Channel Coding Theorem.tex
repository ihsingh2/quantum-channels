\section{Schumacher’s Quantum Noiseless Channel Coding Theorem}

\subsection{Classical Context: Shannon's Source Coding Theorem}
Shannon's Source Coding Theorem provides the foundation for \textbf{lossless data compression}, ensuring data can be encoded efficiently without information loss. According to this theorem:
\begin{itemize}
    \item In the limit, as the length of a stream of i.i.d. random variable data tends to infinity, it is impossible to compress data such that the code rate (average bits per symbol) is less than the entropy of the source.
    \item The average number of bits needed per symbol is given by the source's entropy \( H(X) \).
\end{itemize}

\subsection{Schumacher’s Compression}
In the quantum domain, we use \textbf{qubits}, which can exist in superpositions of states.

\subsubsection{Quantum States and Density Matrices}
A quantum state can be represented by a \textbf{density matrix} \( \rho \), particularly when dealing with mixed states.

For a source emitting states \( \{ |\psi_i\rangle \} \) with probabilities \( \{ p_i \} \), the density matrix \( \rho \) is given by:
\begin{equation}
    \rho = \sum_i p_i |\psi_i\rangle \langle\psi_i|
\end{equation}

\subsubsection{Von Neumann Entropy}
The von Neumann entropy, \( S(\rho) \), is the quantum analogue of Shannon entropy. It measures the uncertainty or the amount of quantum information in \( \rho \) and is defined as:
\begin{equation}
    S(\rho) = -\text{Tr}(\rho \log \rho)
\end{equation}

\section{Statement of Schumacher’s Quantum Noiseless Channel Coding Theorem}

\begin{enumerate}
    \item For a quantum source emitting states described by a density matrix \( \rho \) with entropy \( S(\rho) \), the minimum number of qubits required to encode these states can be compressed to \( n \cdot S(\rho) \) qubits for \( n \) copies of the state.
    \item As \( n \to \infty \), it is possible to compress the data with nearly perfect fidelity, occupying a subspace of dimension \( 2^{n S(\rho)} \).
\end{enumerate}

This means we can compress the information carried by \( n \) copies of \( \rho \) into a smaller subspace without significant information loss.

\section{Proof Outline for Schumacher’s Theorem}

Schumacher’s theorem relies on the concept of a \textbf{typical subspace} within the Hilbert space of the system, similar to how Shannon's theorem relies on typical sequences.

\begin{enumerate}
    \item \textbf{Typical Subspace Formation}: For a large number \( n \) of quantum states, we can identify a subspace of the Hilbert space that is typical for most of the states generated by the source, with dimension approximately \( 2^{n S(\rho)} \).
    \item \textbf{High Probability in the Typical Subspace}: For \( n \) copies of the quantum state \( \rho \), almost all the probability mass of the state distribution is concentrated within this typical subspace, and the probability of the \( n \)-state system lying within it approaches 1 as \( n \) increases.
    \item \textbf{Compression to the Typical Subspace}: The theorem states that we can compress our \( n \)-copy quantum state into a subspace of dimension \( 2^{n S(\rho)} \), thereby reducing the number of qubits from \( n \) to \( n \cdot S(\rho) \). This achieves high fidelity, with minimal information loss for large \( n \).
\end{enumerate}
