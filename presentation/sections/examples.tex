\section{Examples of Quantum Channels}

\begin{frame}{Examples of quantum channels - Bit Flip channel}
    We present the \textbf{Bit Flip} channel.\\
    The Bit flip channel, as its name suggests, flips a qubit from $|0\rangle $ to $|1\rangle $ (or $|1\rangle $ to $|0\rangle $) with probability $p$, and
    does nothing with probability $1-p$:
    \begin{equation}
        \mathcal{N}_p(\rho) = (1-p)\rho + pX\rho X = K_0 \rho K_0^\dagger + K_1 \rho K_1^\dagger
    \end{equation}
    Where $X$ is the Pauli X operator:
    \begin{math}
        X = \begin{bmatrix}
            0 & 1\\
            1 & 0
        \end{bmatrix}
    \end{math}\\
    and $K_0 = (\sqrt{1-p})\mathbb{I} $ and $K_1 = (\sqrt{p})X$ are the Kraus operators.\\

    More channel examples will be produced soon along with analyses on the various capacities of these channels.
\end{frame}

\begin{frame}{Example - Erasure channel}
    % \item[Erasure Quantum Channel]{
        The erasure quantum channel $\mathcal{N}_p$ "erases" the input state $\rho$
        with probability $p$ or transmits the state unchanged with probability $(1−p)$
        \begin{equation}
            \mathcal{N}_p(\rho) \rightarrow (1-p)\rho + (p| e \rangle\langle e |)
        \end{equation}
        where $|e\rangle$ is the "erasure state".
        The classical capacity of this channel is given by:
        \begin{equation}
            C(\mathcal{N}_p) = (1-p)\log (d)
        \end{equation}
        Here $d$ is the dimension of the input system. This demonstrates that the channel
        can transmit some classical information only for $0 \leq p < 1$, otherwise the
        classical capacity is $0$.
        
        On the other hand, its quantum capacity is:
        \begin{equation}
            Q(\mathcal{N}_p) = (1-2p)\log (d)
        \end{equation}
        This is similar to classical capacity, except that information can only be transferred
        if $0 \leq p < 1/2$.
    % }
\end{frame}
\begin{frame}{Example - Phase Erasure Channel}
    % \item[Phase Erasure channel] {
        The Phase Erasure channel "erases" (as the name indicates) the phase of its inputs with
        probability $p$ but does not affect the amplitude. The map is expressed as:
        \begin{equation}
            \mathcal{N}(\rho) \rightarrow (1-p)\rho \otimes |0\rangle\langle 0| + p\frac{\rho + Z\rho Z^\dagger}{2}\otimes |1\rangle\langle 1|
        \end{equation}
        The classical capacity of this channel is:
        \begin{equation}
            C(\mathcal{N}) = 1
        \end{equation}
        The quantum capacity of this channel is:
        \begin{equation}
            Q(\mathcal{N}) = (1-p)\log (d)
        \end{equation}
        where the variables hold the same meaning as the last example.
    % }
\end{frame}

\begin{frame}{Superadditivity}
    Graeme Smith and Jon Yard's example:

    A private Horodecki channel and a Symmetric channel may be combined to
    give a quantum channel with non-zero Quantum capacity, even though the
    individual channels have zero quantum capacity individually.
\end{frame}