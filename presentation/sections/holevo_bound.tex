\section{Holevo Bound}

\begin{frame}{Holevo Bound}
The Holevo bound is the upper bound of information that can be extracted from a quantum system, characterized by:
\[
\chi = S(\rho) - \sum_i p_i S(\rho_i).
\]
\end{frame}

\begin{frame}{Derivation of the Holevo Bound}
To prove this, consider a quantum system that encodes classical information represented by a distribution with probability \( p_x \) for each input \( x \), allowing us to define the classical state \( \rho_x \) as:
\[
\rho_X := \sum_x p_x |x\rangle \langle x|,
\]
where \( |x\rangle \) (ket x) represents a quantum state and \( \langle x| \) (bra x) represents the Hermitian conjugate of \( |x\rangle \).

Since each input is mapped to a quantum state \( \rho_x \), the combined state can be written as:
\[
\rho_{XQ} := \sum_x p_x |x\rangle \langle x| \otimes \rho_x.
\]
The received combined state is represented as:
\[
\rho := \text{tr}_X(\rho_{XQ}) = \sum_x p_x \rho_x,
\]
where \( \text{tr}_X \) represents the trace over \( \rho_{XQ} \).
\end{frame}

\begin{frame}{Derivation of the Holevo Bound}
To bound the maximum obtainable information, we need to bound the mutual information \( S(X : Y) \) with \( S(X : Q) \). From the monotonicity of quantum mutual information, we have:
\[
S(X : Q') \leq S(X : Q),
\]
and similarly:
\[
S(X : Y) \leq S(X : Q'Y).
\]

Combining these inequalities, we get:
\[
S(X : Y) \leq S(X : Q).
\]

Now, we simplify \( S(X : Q) \) as follows:
\begin{align*}
S(X : Q) &= S(X) + S(Q) - S(XQ) \\
         &= S(X) + S(\rho) + \text{tr}(\rho_{XQ} \log \rho_{XQ}) \\
         &= S(\rho) + \sum_x p_x \, \text{tr}(\rho_x \log \rho_x) \\
         &= S(\rho) - \sum_x p_x S(\rho_x),
\end{align*}
which gives the Holevo bound.
\end{frame}
