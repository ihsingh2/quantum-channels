\subsection{Holevo Bound}

The Holevo bound is a fundamental result in quantum information theory that quantifies the maxiumum amount of classical information that can be extracted from a quantum system.
\[
\chi = S(\rho) - \sum_i p_i S(\rho_i).
\]
The proof for this is as follows:

Consider a quantum system that encodes classical information, which can be represented by a distribution with probability $p_x$ for each input $x$. This allows us to define the classical state $\rho_x$ as:
\[
\rho_X := \sum_x p_x \lvert x \rangle \langle x \rvert,
\]
(where $\lvert x \rangle$ (ket $x$) represents a quantum state and $\langle x \rvert$ (bra $x$) represents the Hermitian conjugate of $\lvert x \rangle$.)

Since each input is mapped to a quantum state $\rho_x$, the combined state can be written as:
\[
\rho_{XQ} := \sum_x p_x \lvert x \rangle \langle x \rvert \otimes \rho_x,
\]
where $\rho_x$ represents the density matrix of the quantum system.

The received combined state is represented as:
\[
\rho := \text{Tr}_X (\rho_{XQ}) = \sum_x p_x \rho_x,
\]
where $\text{Tr}_X$ represents the trace of $\rho_{XQ}$.

To bound the maximum information obtainable, we need to bound the mutual information $I(X : Y)$ with $I(X : Q)$, where $X$ is the input, $Y$ is the output, and $Q$ is the quantum state. From the monotonicity of quantum mutual information, quantum mutual information does not increase under quantum operations. Hence, we get:
\[
I(X : Q'Y) \leq I(X : Q).
\]
Similarly,
\[
I(X : Y) \leq I(X : Q'Y).
\]
From these two inequalities, we get:
\[
I(X : Y) \leq I(X : Q).
\]

Simplifying $S(X : Q)$, we have:
\[
I(X : Q) = S(X) + S(Q) - S(XQ),
\]
\[
= S(X) + S(\rho) - \text{Tr}(\rho_{XQ} \log \rho_{XQ}),
\]
\[
= S(\rho) - \sum_x p_x S(\rho_x).
\]
Thus, we arrive at the Holevo bound.

