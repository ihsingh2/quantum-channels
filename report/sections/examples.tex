\section{Examples of Quantum Channels}

\subsection{Capacity}

We provide two examples of quantum channels here, and calculate various metrics of
their capacity:



\subsection{Superadditivity}

A fascinating example of superadditivity in quantum channels involves the concept
of \textbf{superactivation}. Superactivation refers to the phenomenon where two
quantum channels, each with zero individual quantum capacity, can achieve a positive
quantum capacity when used together. This behavior is counterintuitive, and quite
the opposite of what you would expect if we were dealing with classical channels
instead. We provide an example here:

\begin{quotation}
    In a 2008 study by Graeme Smith and Jon Yard \cite{Smith_2008}, the authors demonstrated
    the existence of two zero-capacity quantum channels, each individually incapable of
    transmitting quantum information, which, when used jointly, could support the reliable
    transmission of quantum information. The study highlights that certain quantum channels,
    despite having no individual capacity to send quantum data, can "activate" each other
    under joint use, allowing information to pass through the combined channel setup. This
    unique behavior is not observed in classical communication channels, where combining
    two zero-capacity channels will always result in zero capacity for the combined
    channel.

    Smith and Yard's example involved two specific types of quantum channels known as
    \textbf{Private Horodecki channels} and \textbf{Symmetric channels}. Private Horodecki
    channels are designed to have zero quantum capacity while still exhibiting some level
    of classical and private capacities. Symmetric channels, on the other hand, have zero
    capacity for both classical and quantum transmission due to symmetry constraints that
    prevent any meaningful transmission of information (transmission through the symmetric
    channel would violate the No-cloning theorem). When these two types of channels are
    used together, however, they demonstrate a form of superadditivity: the combined
    quantum channel has a positive capacity, enabling it to transmit quantum information
    reliably.

    The mechanics behind superactivation stem from quantum interactions between the two
    channel types. The private Horodecki channel alone cannot maintain quantum coherence
    due to limitations related to its structure, while the symmetric channel alone is
    constrained by its symmetries. However, when these channels are combined in a
    particular way, each channel’s limitations effectively "cancel out" the limitations of
    the other, allowing a positive quantum capacity to emerge.
\end{quotation}

This example ilustrates an astounding result in Quantum Information Theory. Not only is
it possible for quantum channels to have superadditive capacities, but it is also possible
for channels with Zero capacities to combine to form channels with Non-Zero capacities!

This bizzare result is something worth pondering upon, and in our opinion, the perfect food
for thought to end this Term paper on.
