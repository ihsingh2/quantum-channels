\section{Introduction}

Quantum channels, fundamental in the study of quantum information theory,
describe how quantum states evolve under various types of noise and interaction.
These channels serve as models for real-world scenarios where quantum systems
are exposed to external environments, resulting in state transformations that
are often non-ideal. Quantum channels are central to understanding and developing
quantum communication, computation, and cryptography, as they capture the
limitations and potential of transferring quantum information through noisy or
imperfect media.

In quantum information theory, channels are typically categorized based on their
capacities to transmit information. Just as in classical information theory, where
the Shannon capacity limits the rate at which information can be reliably transmitted,
quantum channels are analyzed through various capacity measures, each capturing
different facets of information transfer. The primary capacities of interest include
\textbf{quantum capacity}, \textbf{classical capacity}, and \textbf{private capacity},
each reflecting the channel’s ability to transmit quantum information, classical
information, or private (secure) information, respectively. These capacities vary
depending on the nature and severity of the noise introduced by the channel, making
it essential to understand the characteristics and limitations of each type.

One of the distinctive aspects of quantum channels is that their capacities are not
always additive, a property that contrasts with classical channels. The superadditivity
phenomenon, where multiple channel uses jointly yield a higher capacity than the sum of
individual capacities, adds a layer of complexity to the analysis. For example,
entanglement-assisted strategies can enhance both classical and quantum transmission rates,
exploiting the channel's underlying quantum mechanics to achieve greater efficiency. This
superadditivity highlights the advantage of leveraging quantum properties like entanglement
and coherence for information transmission.

Our article explores the nature of quantum channels, and delves into the mathematical
frameworks used to quantify their capacities. We discuss the fundamental concepts underlying
quantum channel capacities, analyze the theoretical limits of information transmission, and
cement theses ideas by taking a suitable example.

Before we begin our discussion on channel capacities though, we need to first define some
basic concepts in Quantum Information Theory. Since Quantum Systems work in a fundamentally
different way compared to their classical counterparts, we must define new quantities and
constructs to take these into account.

\subsection{Trace}
It is helpful for us to introduce the \textbf{Trace} of a square operator. It is defined as:
\begin{equation}
    Tr[A] \equiv \displaystyle\sum_{i} \langle i | A | i \rangle
\end{equation}
where $A$ is an operator in the Hilbert space $\mathcal{H}$ and $\{|i\rangle\}$ is a complete
orthonormal basis in $\mathcal{H}$.
We note the following properties about the Trace:
\begin{itemize}
    \item Trace is Linear: $Tr[aA + bB] = aTr[A] + bTr[B]$ where $a$ and $b$ are scalars
    \item Trace is independent of the choice of the basis, as long as it is complete and orthonormal
    \item Trace is Cyclic: $Tr[ABC] = Tr[CAB]$
\end{itemize} 

DEFINE PARTIAL TRACE

\subsection{States}

In quantum mechanics, the state of a system encapsulates all the information needed to describe
it and predict the outcomes of measurements. For a simple, isolated quantum system, the state
is typically represented as a \textbf{state vector} (or \textbf{ket}) \( |\psi\rangle \) in a
complex Hilbert space \( \mathcal{H} \). This vector describes a \textbf{pure state}, where the
system is in a specific, well-defined quantum state. Mathematically, pure states satisfy
\( \langle \psi | \psi \rangle = 1 \), indicating that the vector is normalized. For example,
in a two-level quantum system (qubit), a general pure state is represented as
\( |\psi\rangle = \alpha |0\rangle + \beta |1\rangle \), where \( \alpha \) and \( \beta \) are
complex coefficients satisfying \( |\alpha|^2 + |\beta|^2 = 1 \).

Moreover, it is important to note that the trace of the square of a pure state is 1, or
$Tr[| \psi \rangle^2] = 1$.

However, in many real-world scenarios, systems are not isolated and can exist in \textbf{mixed states},
where they are probabilistically distributed over multiple possible states. Mixed states are described
by \textbf{density operators}, which generalize the concept of quantum states to incorporate
probabilistic mixtures and interactions with external environments. Trace of squares of mixed states is
not necessarily 1. $\rho = 0.5 |0 \rangle \langle 0| + 0.5 |1 \rangle \langle 1|$ is an example of such
a mixed state.

\subsection{Density Operators}

As described above, the \textbf{density operator} (or \textbf{density matrix})
is a mathematical tool that describes the state of a quantum system. Unlike a pure state, which
represents a system with complete certainty about its state vector, the density operator allows
for the representation of mixed states, where a system may exist in a probabilistic mixture of
several possible states. This flexibility makes density operators essential for describing
real-world quantum systems, particularly when dealing with noise, entanglement, and decoherence
effects in quantum channels. The density operator for a system encodes all observable properties
of the system.

Mathematically, a density operator $\rho$ corresponding to the ensemble $\mathcal{E} \equiv
\{p_X(x), |\psi\rangle \}_{x \in \mathcal{X}}$ is defined as:
\begin{equation}
    \rho \equiv \displaystyle\sum_{x \in \mathcal{X} } p_X(x) |\psi \rangle \langle \psi|
\end{equation}
This definition of density operators can also be interpreted as the state of the system. One
more way of understanding the density operator is to think of it as the expected state:
\begin{equation}
    \rho = \mathbb{E}_X \{ |\psi \rangle \langle \psi| \}
\end{equation}
The definition gives us a wide picture of the Density operator. As a consequence of the above,
the Density operator has certain properties that it always satisfies. These are summarized here
as follows:
\begin{itemize}
    \item Density operator has unit trace: $Tr[\rho] = 1$
    \item Density operator is Hermitian: $\rho^\dagger = \rho$
    \item Density operators are positive semi-definite, meaning $\langle \varphi  | \rho | \varphi \rangle
    \forall \varphi$ which may be written as $\rho >= 0$
\end{itemize}
It is worth mentioning that every ensemble has a unique density operator, but the opposite
is not necessarily true, and the same density operator could correspond to multiple ensembles.

\subsection{Environment and Reference Systems}

In studying quantum systems, we often introduce reference systems and environments to clarify
how entanglement, decoherence, and information transfer occur. A reference system $R$ is often
used as an ancillary(additional) system that helps track information about another system $A$.
Reference systems are useful for defining quantities like coherent information and entanglement.
By examining how the information in $A$ relates to the reference $R$, we gain insights into the
entanglement properties and capacities of quantum channels or systems.

In realistic scenarios,
no quantum system is completely isolated. An environment $E$ represents everything outside the
system of interest, which may interact with it and cause decoherence—the process by which a quantum
system loses its coherence and behaves more classically. When analyzing quantum systems, we model
the system and its environment as a combined entity as mixed states that allow us to study how
noise and information loss occur in quantum channels.

A reference system is often modelled as a \textbf{Product state}:
\begin{definition}[Product State]
    The Tensor Product of a state with another state (often the environment or a reference system).
    \begin{equation}
        \rho_{AB} = \rho_A \otimes \rho_B
    \end{equation}
\end{definition}
The product state models all the interactions between the system and the environment. Besides modelling
environment interaction, product states can also model interaction between two systems and give
information about the correlation between them.
\begin{definition}[Bipartite states]
    A state is said to be Bipartite if it is a Tensor Product of two different states.
\end{definition}

This brings us to the related definition of separable states and entanglement:
\begin{definition}[Separable states]
    A bipartite state $\sigma_{AB}$ is said to be Separable if it can be written in the form:
    \begin{equation}
        \sigma_{AB} = \displaystyle\sum_{x} p_X(x)|\psi_x\rangle\langle\psi_x|_A \otimes |\phi_x\rangle\langle\phi_x|_B
    \end{equation}
    for some probability distribution $p_X(x)$ and sets $\{| \psi_x \rangle_A\}$ and $\{| \phi_x \rangle_B\}$ of pure states.
\end{definition}

\begin{definition}[Entanglement]
    A state that is not separable is said to be entangled.
\end{definition}

IMAGES AND THE 3 STEP PROCESS

\subsection{Channels as CPTP maps}

Now we have established most things required to begin our major discussion on Quantum Channels. Any
Quantum channel is a model of a physical transformation. Hence, there are certain rules that such a
channel must follow in order to make physical sense.

\subsection{Choi-Kraus Representation and the Choi operator}

The above definition (CPTP maps) of the quantum channel leads to a certain representation that turns
out to be incredibly useful and insightful. We summarize this representation as the Choi-Kraus theorem:
\begin{theorem}[Choi-Kraus Theorem]
    Any map $\mathcal{N}_{A\rightarrow B}:\mathcal{L}(\mathcal{H}_A) \rightarrow \mathcal{L}(\mathcal{H}_B)$
    (where $\mathcal{L}(\mathcal{H}_A)$ is the space of all Linear operators on $\mathcal{H}_A$) is Linear,
    Completely Positive and Trace Preserving if and only if it has a Choi-Kraus decomposition as follows:
    \begin{equation}
        \mathcal{N}_{A \rightarrow B}(X_A) = \displaystyle\sum_{l=0}^{d-1} V_l X_A V_l^\dagger
    \end{equation}
    where $X_A \in \mathcal{L}(\mathcal{H}_A)$, $V_l \in \mathcal{L}(\mathcal{H_A}, \mathcal{H_B})$ 
    $\forall l \in \{0,\dots,d-1\}$ with $d < dim(\mathcal{H}_A)dim(\mathcal{H}_B)$ and
    \begin{equation}
        \displaystyle\sum_{l=0}^{d-1} V_l^\dagger V_l = I_A
    \end{equation}
\end{theorem}

\subsection{POVM Measurements}