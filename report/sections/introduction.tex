\section{Introduction}

Quantum channels, fundamental in the study of quantum information theory,
describe how quantum states evolve under various types of noise and interaction.
These channels serve as models for real-world scenarios where quantum systems
are exposed to external environments, resulting in state transformations that
are often non-ideal. Quantum channels are central to understanding and developing
quantum communication, computation, and cryptography, as they capture the
limitations and potential of transferring quantum information through noisy or
imperfect media.

In quantum information theory, channels are typically categorized based on their
capacities to transmit information. Just as in classical information theory, where
the Shannon capacity limits the rate at which information can be reliably transmitted,
quantum channels are analyzed through various capacity measures, each capturing
different facets of information transfer. The primary capacities of interest include
\textbf{quantum capacity}, \textbf{classical capacity}, and \textbf{private capacity},
each reflecting the channel’s ability to transmit quantum information, classical
information, or private (secure) information, respectively. These capacities vary
depending on the nature and severity of the noise introduced by the channel, making
it essential to understand the characteristics and limitations of each type.

One of the distinctive aspects of quantum channels is that their capacities are not
always additive, a property that contrasts with classical channels. The superadditivity
phenomenon, where multiple channel uses jointly yield a higher capacity than the sum of
individual capacities, adds a layer of complexity to the analysis. For example,
entanglement-assisted strategies can enhance both classical and quantum transmission rates,
exploiting the channel's underlying quantum mechanics to achieve greater efficiency. This
superadditivity highlights the advantage of leveraging quantum properties like entanglement
and coherence for information transmission.

Our article explores the nature of quantum channels, and delves into the mathematical
frameworks used to quantify their capacities. We discuss the fundamental concepts underlying
quantum channel capacities, analyze the theoretical limits of information transmission, and
cement theses ideas by taking a suitable example.

Before we begin our discussion on channel capacities though, we need to first define some
basic concepts in Quantum Information Theory. Since Quantum Systems work in a fundamentally
different way compared to their classical counterparts, we must define new quantities and
constructs to take these into account.

\subsection{States}

In quantum mechanics, the state of a system encapsulates all the information needed to describe
it and predict the outcomes of measurements. For a simple, isolated quantum system, the state
is typically represented as a \textbf{state vector} (or \textbf{ket}) \( |\psi\rangle \) in a
complex Hilbert space \( \mathcal{H} \). This vector describes a \textbf{pure state}, where the
system is in a specific, well-defined quantum state. Mathematically, pure states satisfy
\( \langle \psi | \psi \rangle = 1 \), indicating that the vector is normalized. For example,
in a two-level quantum system (qubit), a general pure state is represented as
\( |\psi\rangle = \alpha |0\rangle + \beta |1\rangle \), where \( \alpha \) and \( \beta \) are
complex coefficients satisfying \( |\alpha|^2 + |\beta|^2 = 1 \).

However, in many real-world scenarios, systems are not isolated and can exist in \textbf{mixed states},
where they are probabilistically distributed over multiple possible states. Mixed states are described
by \textbf{density operators}, which generalize the concept of quantum states to incorporate
probabilistic mixtures and interactions with external environments.

\subsection{Density Operators}

In quantum mechanics and quantum information theory, the \textbf{density operator} (or \textbf{density matrix})
is a mathematical tool that describes the state of a quantum system. Unlike a pure state, which
represents a system with complete certainty about its state vector, the density operator allows
for the representation of mixed states, where a system may exist in a probabilistic mixture of
several possible states. This flexibility makes density operators essential for describing
real-world quantum systems, particularly when dealing with noise, entanglement, and decoherence
effects in quantum channels. Mathematically, a density operator \( \rho \) for a system is a
positive semi-definite, Hermitian operator with trace 1, which encodes all observable properties
of the system.

The density operator $\rho$ corresponding to the ensemble $\mathcal{E} \{\equiv p_X(x), |\psi\rangle \}$
is defined as:
\begin{equation}
    \rho \equiv \displaystyle\sum_{x \in \mathcal{X} } p_X(x) |\psi \rangle \langle \psi|
\end{equation}
This definition of density operators can also be interpreted as the state of the system.

\subsection{Trace}

\subsection{Environment and Reference Systems}

\subsection{Positivity and Complete Positivity}

\subsection{Channels as CPTP maps}

\subsection{Kraus Representation}

\subsection{POVM Measurements}