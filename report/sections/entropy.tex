\section{Quantum(Von Neumann) Entropy}
Von Neumann entropy is the measure of disorder of a quantum state, which can be calculated as:
\[
S(\rho) = -\text{Tr}(\rho \log \rho),
\]
where $\text{Tr}$ is the trace of the density matrix. It can also be written as:
\[
S(\rho) = -\sum_i \lambda_i \log \lambda_i,
\]
where $\lambda_i$ is an eigenvalue of the density matrix $\rho$.

For example, a quantum system with the mixed state:
\[
\rho = 
\begin{pmatrix}
0.8 & 0 \\
0 & 0.2
\end{pmatrix}
\]
has eigenvalues $0.8$ and $0.2$ (since the quantum state is diagonal, the eigenvalues are simply the diagonal elements). The von Neumann entropy of this system is:
\[
S(\rho) = -\sum_i \lambda_i \log \lambda_i = -(0.8 \log 0.8 + 0.2 \log 0.2) = 0.3974.
\]

\section{Properties of Quantum Entropy}

\subsection{Non-Negativity}
The quantum entropy $H(\rho)$ is non-negative for any density operator $\rho$:
\[
H(\rho) \geq 0.
\]
\subsubsection{Proof:}
The proof of this follows from the non-negativity of Shannon entropy.

\subsection{Minimum Value}
The minimum value of $H(\rho)$ is zero, which occurs when the density operator $\rho$ is in its pure state.

\subsubsection{Proof:}
When the eigenvalues of the density operator are distributed such that all the probability mass is concentrated on one eigenvector (and the other eigenvalues are zero), the density operator has a rank of one. This corresponds to its pure state.

\subsection{Maximum Value}
The maximum value of $H(\rho)$ is $\log d$, where $d$ is the dimension of the system. This occurs when the density operator is in its maximally mixed state.

\subsubsection{Proof:} 
The proof of this property is the same as in the classical case. A maximally mixed state distributes the probability mass equally among all eigenvalues, achieving the maximum entropy.

\subsection{Concavity}
Let $\rho_x$ $\in$ D(H) and let $p_x(x)$ be the probability distribution. The entropy is concave in the density operator:
\[
H(\rho) \geq \sum_x p_X(x) H(\rho_x),
\]
where $\rho \equiv \sum_xp_x(x)\rho_x$

% \subsection*{Isometric Invariance}

\section{Joint Quantum Entropy}
The joint quantum entropy H(AB)$_\rho$ of the density operator $\rho_{AB} \in \mathcal{D}(\mathcal{H_A} \otimes \mathcal{H_B}) $ for a bipartite system AB follows naturally from the definition of quantum entropy
\[
H(AB)_\rho \equiv -Tr\{\rho_{AB} \log \rho_{AB}\},
\]


\section{Conditional Quantum Entropy}
Let $\rho_{AB} \in D(H_A \otimes H_B)$. The conditional quantum entropy $H(A|B)_\rho$ of $\rho_{AB}$ is equal to the difference of the joint quantum entropy $H(AB)_\rho$ and marginal entropy $H(B)_\rho$
\[
H(A|B)_\rho \equiv H(AB)_\rho - H(B)_\rho,
\]

\subsection{Conditioning does not increase entropy}
Consider a bipartite quantum state $\rho_{AB}$. Then the following applies for marginal entropy $H(A)_\rho$ and conditional quantum entropy $H(A|B)_\rho$:
\[
H(A)_\rho \geq H(A|B)_\rho
\]
The proof for this property is the same as the classical one


\section{Quantum Mutual Information}
the quantum mutual information of a bipartite state $\rho_{AB} \in \mathcal{D}(\mathcal{H_A} \otimes \mathcal{H_B})$ is defined as follows:
\[
I(A;B)_\rho = H(A)_\rho + H(B)_\rho - H(AB)_\rho.
\]
The following relations hold for quantum mutual information the same as for classical mutual information:
\[
I(A;B)_\rho = H(A)_\rho - H(A|B)_\rho.
\]
\[
            \quad = H(B)_\rho - H(B|A)_\rho
\]
These formulae lead to the following relations between quantum mutual information and coherent information:
\[
I(A;B)_\rho = H(A)_\rho - I(A\rangle{}B)_\rho
\]
\[
 = H(B)_\rho - I(B\rangle{}A)_\rho
\]

\subsection{Non-negativity of Quantum Mutual Information}
The quantum mutual information of any bipartite quantum state $\rho_{AB}$ is non-negative.
\[
I(A;B)_\rho \geq 0
\]


\section{Conditional Quantum Mutual Information}
We define the conditional quantum mutual information $I(A;B|C)_\rho$ of any tripartite state $\rho_{ABC} \in \mathcal{D}(\mathcal{H_A} \otimes \mathcal{H_B} \otimes \mathcal{H_C})$ similarly to how we did in the classical one:
\[
I(A;B|C)_\rho \equiv H(A|C)_\rho + H(B|C)_\rho - H(AB|C)_\rho
\]

\subsection{Chain rule}
The CQMI obeys a chain rule:
\[
I(A;BC)_\rho = I(A;B)_\rho + I(A;C|B)_\rho
\]
The proof for this follows the classical one


\subsection{Non-Negativity if CQMI}
Let $\rho _{ABC} \equiv \mathcal{D}(\mathcal{H_A} \otimes \mathcal{H_B} \otimes \mathcal{H_C}))$.Then the CQMI is non-negative:
\[
I(A;B|C)_\rho \geq 0.
\]

\section{Quantum Relative Entropy}
The quantum relative entropy $D(\rho||\sigma)$ between a density operator $\rho \in D(H)$ and a positive semi-definite operator $\sigma \in L(H)$ is defined as follows:
\[
D(\rho \Vert \sigma) = Tr\{\rho[log\rho - log\sigma]\}
\]
if the following condition is met:
\[
supp(\rho) \subseteq supp(\sigma)
\]
Else it is defined as +$\infty$
\newline
\newline

\subsection{Monotonicity of Quantum Relative Entropy}
Let $\rho \in \mathcal{D}(\mathcal{H}), \sigma \in \mathcal{L}(\mathcal{H}) \text{ and } \mathcal{N}:\mathcal{L}(\mathcal{H}) \rightarrow \mathcal{L}(\mathcal{H}\prime)$ be a quantum channel. The quantum relative entropy can only decrease or stay the same if we apply the same quantum channel N to $\rho$ and $\sigma$:
\[
D(\rho||\sigma) \geq D(N(\rho)||N(\sigma))
\]
This implies non-negativity of quantum relative entropy in some cases

\subsection{Non-negativity}
Let $\rho \in \mathcal{D}(\mathcal{H})$, and let $\sigma \in \mathcal{L}(\mathcal{H})$ be positive semi-definite such that $Tr\{\rho\} \leq 1$. Then the quantum relative entropy $D(\rho||\sigma)$ is non-negative:
\[
D(\rho||\sigma) \geq 0
\]
and $D(\rho||\sigma) = 0$ if and only if $\rho = \sigma$
\newline
\newline
Proof:
The first part of this proof follows from the monotonicity of relativity
\[
D(\rho||\sigma) \geq D(Tr\{\rho\}||Tr\{\sigma\} = Tr\{\rho\}log\left(\frac{Tr\{\rho\}}{Tr\{\sigma\}}\right) \geq 0
\]
If $\rho = \sigma$, then we get that $D(\rho||\sigma)$ = 0.This implies that the inequality above is saturated and thus Tr\{$\rho$\} = Tr\{$\sigma$\} = 1. Let M be an arbitrary measurement channel. From the monotonicity of relative entropy we can conclude that $D(M(\rho)||M(\sigma)$ = 0. The equality condition for non-negative entropy gives us $M(\rho) = M(\sigma)$. Since this equality holds for any possible measurement channel, we can conclude that $\rho = \sigma$
\newline

\subsection{Property}
Let $\rho \in \mathcal{D}(\mathcal{H})$ and $\sigma, \sigma' \in \mathcal{L}(\mathcal{H})$ be positive semi-definite. Suppose that $\sigma \leq \sigma'$. Then
\[
D(\rho \| \sigma') \leq D(\rho \| \sigma).
\]


Proof:
The assumption that $\sigma \leq \sigma'$ is equivalent to $\sigma' - \sigma$ being positive semi-definite. Then the following operator is positive semi-definite: 
\[
\sigma \otimes \ket{0}\bra{0}_X + (\sigma' - \sigma) \otimes \ket{1}\bra{1}_X,
\]
and as a consequence
\[
D(\rho \| \sigma) = D(\rho \otimes \ket{0}\bra{0}_X \| \sigma \otimes \ket{0}\bra{0}_X + (\sigma' - \sigma) \otimes \ket{1}\bra{1}_X), \tag{11.141}
\]
which follows by a direct calculation (essentially the same reasoning as that used to solve Exercise 11.8.8). By monotonicity of quantum relative entropy (Theorem 11.8.1), the quantum relative entropy does not increase after discarding the system $X$, so that
\[
D(\rho \otimes \ket{0}\bra{0}_X \| \sigma \otimes \ket{0}\bra{0}_X + (\sigma' - \sigma) \otimes \ket{1}\bra{1}_X)
\geq D(\rho \| [\sigma + (\sigma' - \sigma)]) = D(\rho \| \sigma'),
\]
concluding the proof.
