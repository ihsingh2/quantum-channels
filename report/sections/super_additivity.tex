\section{Super Additivity of Quantum Channels}

Superadditivity in quantum information theory refers to the phenomenon where
certain types of quantum capacities increase when multiple channels (or multiple
uses of the same channel) are considered, rather than treating each channel
(or use) independently. This property arises because quantum entanglement and
correlations can enhance the effectiveness of the channel when used in a
collective or joint manner. Superadditivity is a fundamental property that
distinguishes quantum information theory from classical information theory.

This Superadditivity is the result of entanglement and other exotic correlation
phenomena that arises in quantum systems. Superadditivity enhances
communication capabilities of quantum channels and allows more communication
than what their classical counterparts could (sometimes even through
zero capacity channels).

\subsection{Classical Capacity}
One manifestation of superadditivity appears in the classical capacity of quantum
channels, particularly through the Holevo information. As we have seen before, the
Holevo Information of a quantum channel is an upper bound on the classical information
that can be transmitted through it. Also as we have seen before, Classical capacity
has been defined in terms of the Holevo information. As a result, it can exhibit
superadditivity when multiple channel uses are considered together. For example, when
encoding information across multiple uses of the channel in an entangled way or by
using complex encoding schemes, the total classical capacity often exceeds what would be
achievable by simply using each channel separately. Thus, superadditivity allows us to
exploit collective encoding and decoding strategies to maximize the amount of classical
information extracted from quantum channels.

Although it is worth noting that we don't yet know the precise mathematics behind why
Holevo information is superadditive, we just know from counter examples that additivity
is not the general case and super additivity can exist.

Mathematically, we write the Holevo information of two channels acting in parallel as:
\begin{equation}
    \chi(\mathcal{N} \otimes \mathcal{M}) \geq \chi(\mathcal{N}) + \chi(\mathcal{M})
\end{equation}
Now, since classical capacity is:
\begin{equation}
    C(\mathcal{N}) = \lim_{n \rightarrow \infty }\frac{1}{n}\chi(\mathcal{N}^{\otimes n})
\end{equation}
For channel $\mathcal{N}\otimes\mathcal{M}$ we have:
\begin{align}\begin{split}
    C(\mathcal{N}\otimes\mathcal{M}) & = \lim_{n \rightarrow \infty }\frac{1}{n}\chi[(\mathcal{N}\otimes\mathcal{M})^{\otimes n}]\\
    & \geq \lim_{n \rightarrow \infty }\frac{1}{n}[\chi(\mathcal{N}^{\otimes n}) + \chi(\mathcal{M}^{\otimes n})]\\
    & = \lim_{n \rightarrow \infty }\frac{1}{n}\chi(\mathcal{N}^{\otimes n}) + \lim_{n \rightarrow \infty }\frac{1}{n}\chi(\mathcal{M}^{\otimes n})\\
    & = C(\mathcal{N} + \mathcal{M})
\end{split}\end{align}

\subsection{Quantum Capacity}
Coherent Information

\subsection{Private Capacity}